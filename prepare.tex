\section{準備}
\label{sec:prepare}
\subsection{マルコフモデル}
\label{sec:prepare:markov-model}
まず,\textbf{マルコフモデル}について説明をする.
マルコフモデルとは,時間に沿って順に観測されるデータである
\textbf{時系列観測データ}の各データ (\textbf{観測変数}) が,
そのデータが観測される時点$i$よりも
過去のデータに対してのみ従属関係にあるようなデータ列が
従うモデルのことである.
なお,ある時点$i$での観測変数の個数は
1つだけと仮定する.

ある時系列観測データを
$X \defeq \{\bvec{x}_1, \dots, \bvec{x}_N|\bvec{x}_i \in \mathbb{R}^D\}\ (D:\text{Dimension})$

とおき,$N$個のデータ点の同時分布を考える.
これは確率の乗法定理を用いることで,
\begin{align}
  %% \begin{aligned}
    & p(\bvec{x}_1, \dots, \bvec{x}_N) \\
    & = p(\bvec{x}_1)p(\bvec{x}_2, \dots, \bvec{x}_N| \bvec{x}_1) \\
    & = p(\bvec{x}_1)p(\bvec{x}_2| \bvec{x}_1)p(\bvec{x}_3, \dots, \bvec{x}_N| \bvec{x}_1, \bvec{x}_2) \\
    & = \dots \\
    & = p(\bvec{x}_1)\prod_{n=2}^{N}p(\bvec{x}_n| \bvec{x}_{n-1}, \dots, \bvec{x}_1) \label{eq:prob-multiplication}\\
  %% \end{aligned}
\end{align}
と書ける.%なお,式\eqref{eq:prob-multiplication}は
%マルコフモデルでなくても成り立つ.

\subsubsection{1次マルコフモデル}
\label{sec:prepare:first-order-markov-model}
次に,\textbf{1次マルコフモデル}について説明する.
これはある時点$i$での観測変数が,
その1つ前の時点の観測変数にのみ従属であるような
確率モデルである.式\eqref{eq:prob-multiplication}
を参考にして1次マルコフモデルを定式化すると,
\begin{align}
  & p(\bvec{x}_1, \dots, \bvec{x}_N) = p(\bvec{x}_1)\prod_{n=2}^{N}p(\bvec{x}_n|\bvec{x}_{n-1}) \label{eq:markov-model}\\
\end{align}
のように書ける.ここで,式\eqref{eq:markov-model}は
\begin{align}
  p(\bvec{x}_n| \bvec{x}_{n-1}, \dots, \bvec{x}_1) = p(\bvec{x}_n| \bvec{x}_{n-1})\hspace{10mm}(n\in \{1,\dots,N\})
\end{align}
という1次マルコフモデルの特性を定式化して導出した.

\subsection{隠れマルコフモデル}
\label{sec:prepare:hidden-markov-model}
\textbf{隠れマルコフモデル} (Hidden Markov Model, \textbf{HMM})とは,データとして得られる状態
 (観測変数) の他に,
その観測変数がどの状態としてが出力されるかの確率に
影響を与えるが,出力はされないために実際の状態がわからない
離散的な値をとる\textbf{潜在変数}(\textbf{隠れ変数})
というものを導入したマルコフモデルのことである.
以下,潜在変数を$Z \defeq \{\bvec{z}_1, \dots, \bvec{z}_N\}$
と表す.隠れマルコフモデルの具体的な条件を書くと,
\begin{itemize}
\item 観測変数$\bvec{x}_n$は潜在変数$\bvec{z}_n$の影響を受けて出力される
\item 観測変数どうしは独立である
\item 潜在変数どうしは1次マルコフモデルに従い生成される
\end{itemize}
となる.潜在変数の状態の個数である\textbf{状態数}を$K$,
隠れマルコフモデルのパラメータをそれぞれ
$\bvec{\pi}, A, \bvec{\phi}$とおき,
$\theta = \{\bvec{\pi}, A, \bvec{\phi}\}$とおく.
ただし,$\bvec{\pi} = (\pi_1,\dots,\pi_K)^{\mathrm T}$
は潜在変数の初期変数である
$\bvec{z}_1$の状態を決定する確率である
\textbf{初期確率} (←用語は適切か?) を表し,
$A = [A_{jk}]_{K\times K}$は,任意の時点での潜在変数の状態が
$j$のとき,次の時点で潜在変数の状態が$k$に遷移する
確率である\textbf{遷移確率}を表す.
また,$\bvec{\phi}=(\phi_1,\dots,\phi_K)^{\mathrm T}$は任意の時点で潜在変数の状態が
決定したとき,観測状態が出力される確率である
\textbf{出力確率}を表す.
ここで,$\pi_k \defeq p(z_{1k}=1)$, $A_{jk} \defeq p(z_{nk}|z_{n-1,j})$
とする.

このような条件のもとに隠れマルコフモデルを定式化すると
以下のように表せる.
\begin{align}
  p(X,Z|\theta) &= p(\bvec{x}_1, \dots, \bvec{x}_N, \bvec{z}_1, \dots, \bvec{z}_N|\theta) \\
  &= p(\bvec{z}_1|\theta)p(\bvec{x}_1, \dots, \bvec{x}_N, \dots, \bvec{z}_N|\bvec{z}_1,\theta) \\
  &= p(\bvec{z}_1|\bvec{\pi})p(\bvec{x}_1, \dots, \bvec{x}_N,\bvec{z}_2, \dots, \bvec{z}_N|\bvec{z}_1,\theta) \\
  &= p(\bvec{z}_1|\bvec{\pi})\prod_{n=2}^Np(\bvec{z}_n|\bvec{z}_{n-1},\theta)p(\bvec{x}_1, \dots, \bvec{x}_N|\bvec{z}_1,\dots,\bvec{z}_N,\theta) \\
  &= p(\bvec{z}_1|\bvec{\pi})\prod_{n=2}^Np(\bvec{z}_n|\bvec{z}_{n-1},A)p(\bvec{x}_1, \dots, \bvec{x}_N|\bvec{z}_1,\dots,\bvec{z}_N,\theta) \\
  &= p(\bvec{z}_1|\bvec{\pi})\prod_{n=2}^Np(\bvec{z}_n|\bvec{z}_{n-1},A)\prod_{n=1}^Np(\bvec{x}_n|\bvec{z}_n,\theta) \\
  &= p(\bvec{z}_1| \bvec{\pi})\prod_{n=2}^Np(\bvec{z}_n|\bvec{z}_{n-1},A)\prod_{n=1}^Np(\bvec{x}_n|\bvec{z}_n,\bvec{\phi}) \label{eq:hmm-distribution}\\
\end{align}

初期確率$\bvec{\pi}$について,$\pi_k=p(z_{1k}=1)$より,
\begin{align}
  p(\bvec{z}_1|\bvec{\pi}) = \prod_{k=1}^K\pi_k^{z_{1k}} \label{eq:initial-probability}
\end{align}
と表すことができる.
さらに,$A_{jk} = p(z_{nk}|z_{n-1,j})$より,
\begin{align}
  p(\bvec{z}_n|\bvec{z}_{n-1},A) = \prod_{k=1}^K\prod_{j=1}^KA_{jk}^{z_{n-1,j}z_{nk}} \label{eq:transition-probability}
\end{align}
と表せる.

\subsection{数学的準備}
\label{sec:prepare:math-prepare}
各アルゴリズムの説明のために必要な数学的準備を行う.
ある時系列観測データを
$X = \{\bvec{x}_1, \dots, \bvec{x}_N\}$,

その時系列観測データの裏にある潜在変数列を
$Z = \{\bvec{z}_1, \dots, \bvec{z}_N\}$

と表す.ただし,$x_n \in \mathbb{R}^D,\ (n\in \{1,\dots,N\})$とする.
各$\bvec{z}_n$は,
潜在変数の状態数を$K$とすると,
\begin{align}
  &\bvec{z}_n =
  \begin{bmatrix}
    z_{n1} \\
    \vdots \\
    z_{nK} \\
  \end{bmatrix}\\
  & z_{nk} \in \{0,1\}, (\forall n)\ n\in \{1,\dots,N\},
  (\forall k)\ k \in \{0,\dots,K\} \\
  &\sum_{k=1}^Kz_{nk} = 1, n\in \{1,\dots,N\}
\end{align}
を満たすものとする.つまり$\bvec{z}_n$は
その$K$個あるベクトルの成分のうち
どれか1つだけが$1$で残りが$0$であるような
いわゆる1-of-K表現のベクトルである.

これからEMアルゴリズムを用いて隠れマルコフモデルの
パラメータの更新を行うにあたり,
以下の記号を導入する.

\begin{align}
    Q(\theta, \theta^{old}) &\defeq \mathbb{E}[\log p(X, Z| \theta)] = \sum_Zp(Z| \theta^{old})\log p(X, Z| \theta) \\
    \gamma(\bvec{z}_n) &\defeq p(\bvec{z}_n|X, \theta^{old}) \\
    \xi(\bvec{z}_{n-1},\bvec{z}_n) &\defeq p(\bvec{z}_{n-1},\bvec{z}_n|X,\theta^{old}) \\
    \gamma(z_{nk}) &\defeq \mathbb{E}[z_{nk}] = \sum_{\bvec{z}_n}\gamma(\bvec{z}_n)z_{nk} \\
    \xi(z_{n-1,j},z_{nk}) &\defeq \mathbb{E}[z_{n-1,j}z_{nk}] = \sum_{\bvec{z}_{n-1},\bvec{z}_n}\xi(\bvec{z}_{n-1},\bvec{z}_n)z_{n-1,j}z_{nk} \\
\end{align}
ここで,$\theta^{old}$はパラメータの更新を行う前の
古いパラメータを表している.上の各記号について簡単に説明すると,
\begin{itemize}
\item $Q(\theta,\theta^{old})$ \\
  隠れマルコフモデルを表す確率分布$p(X,Z|\theta)$の
  対数尤度関数の期待値
\item $\gamma(\bvec{z}_n)$ \\
  古いパラメータ$\theta^{old}$上での潜在変数$\bvec{z}_n$
  の確率分布
\item $\xi(\bvec{z}_{n-1},\bvec{z}_n)$ \\
  古いパラメータ$\theta^{old}$上での
  潜在変数の組$(\bvec{z}_{n-1},\bvec{z}_n)$の同時確率分布
\item $\gamma(z_{nk})$ \\
  $z_{nk}=1$となる期待値
\item $\xi(z_{n-1,j},z_{nk})$ \\
  $z_{n-1,j}=1,z_{nk}=1$となる期待値
\end{itemize}

である.なお,$\gamma(z_{nk})$,$\xi(z_{n-1,j},z_{nk})$を
\textbf{負担率}と呼ぶ.

%% EOF

\section{その他}
\label{sec:others}
\subsection{未解決}
\label{subsec:others:I-don't-know}
\begin{itemize}
\item 各アルゴリズムの計算量の見積もりの証明
\item Baum-Welchの正規分布の場合の平均と分散の最大化の計算←18/01/16証明完了.
\item $p(\bvec{x}_{n+1},\dots,\bvec{x}_N|\bvec{z}_n)=p(\bvec{x}_{n+1},\dots,\bvec{x}_N)$
  にはなぜならない?←おそらく解決.
  section \ref{subsubsec:others:notice:depend-independ} を参照.\\
\item section \ref{sec:Baum-Welch:abst} Baum-Welch \\
  期待値は全てのパラメータに
  対して上に凸な関数なのか?そうでなければ微分して
  $0$になるところで最大値をとることはいえない.
\end{itemize}
\subsection{気づいたことメモ}
\label{subsec:others:notice}
\subsubsection{独立変数,従属変数}
\label{subsubsec:others:notice:depend-independ}
$p(\bvec{x}_n|\bvec{z}_{n-1},\bvec{z}_n)=p(\bvec{x}_n|\bvec{z}_n)$だが,$p(\bvec{x}_n|\bvec{z}_{n-1})\neq p(\bvec{x}_n)$

↑なぜ?

仮説.時間のあるときにきちんと計算したい.
おそらく条件付き確率の定義から注意深く計算していけば
証明出来る気がする

$\bvec{z}_n$が与えられたときは$\bvec{x}_n$は
$\bvec{z}_{n-1}$に対して独立.$\bvec{x}_n$は
$\bvec{z}_n$によってのみ決定されるから.
ただし,$\bvec{z}_n$がないとき,つまり
$p(\bvec{x}_n|\bvec{z}_{n-1})$のときは$\bvec{x}_n$は
$p(\bvec{z}_{n-1})$に対して従属.
なぜならば $\bvec{z}_n$が $\bvec{z}_{n-1}$に対して
従属であり, $\bvec{x}_n$が $\bvec{z}_n$に対して従属だから.
\subsubsection{Forward-Backward}
\label{subsec:others:notice:fw-bw}
Forward-Backwardは
\begin{itemize}
\item Viterbiと組み合わせて使われる場合と,
\item Baum-Welchと組み合わせて使われる場合
\end{itemize}
の2種類がある.

前者では
まずViterbiをやって,モデルの最大事後確率を求める.
つぎにその$p(Z|X,\theta)$を使って,
各$p(\bvec{z}_n|X,\theta)$を求める.
この場合は$\xi(\bvec{z}_{n-1},\bvec{z}_n)$は求めていない.

後者では
Baum-WelchでEMする際に$\gamma$,\ $\xi$の具体的な値が
必要になるから,そのときにForward-Backwardで計算する.

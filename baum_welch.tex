\section{Baum-Welch algorithm}
\label{sec:Baum-Welch}
\subsection{概要}
\label{sec:Baum-Welch:abst}
Baum-Welch algorithmとは,HMMにおいてEM アルゴリズムを用いて
HMMのパラメータを最適化するアルゴリズムである.
\vspace{\baselineskip}

$\cdot$ 手順
\begin{enumerate}
\item E step \label{enum:e-step} \\
  現在推定されているパラメータに対する,モデルの尤度関数
  $p(X,Z|\theta)$の(対数の)期待値を求める.
\item M step \label{enum:m-step} \\
  E stepで求めた期待値を最大化するようなパラメータを求める.
\item 繰り返し \\
  \ref{enum:e-step}, \ref{enum:m-step}をパラメータの値が
  収束または値の更新時の差分が一定以下になるまで繰り返す.
\end{enumerate}
\subsubsection{入力}
\label{sec:Baum-Welch:input}
%% HMMの確率分布関数$p(X,Z|\theta)$,
観測データ列 $X=\{\bvec{x}_1,\dots,\bvec{x}_N\}$,
初期状態のパラメータ
$\theta^{old} = \{\bvec{\pi}, A, \bvec{\phi}\}$ \\
\hphantom{潜}潜在変数が出力される期待値
$\gamma(z_{nk})$,$\xi(z_{n-1,j},z_{nk})$ 
(EMの更新のたびに必要)
\subsubsection{出力}
\label{sec:Baum-Welch:output}
最尤推定したパラメータ $\theta = \{\bvec{\pi}, A, \bvec{\phi}\}$
\subsection{アルゴリズム}
\label{sec:Baum-Welch:algorithm}
\subsubsection{E step}
\label{sec:Baum-Welch:e-step}
尤度関数の期待値$Q(\theta, \theta^{old})$を
$\gamma(z_{nk})$と$\xi(z_{n-1,j},z_{nk})$の式に変形する.

そのように変形する理由は,
section \ref{sec:Forward-Backward}で
$\gamma(z_{nk})$と$\xi(z_{n-1,j},z_{nk})$
を陽に求めることができるからである.

式\eqref{eq:hmm-distribution}に注意すると,
\begin{align}
  %% \begin{aligned}
    Q(\theta, \theta^{old})& =  \sum_Zp(Z| \theta^{old})\log p(X, Z| \theta) \\
    & =  \sum_Zp(Z| \theta^{old})\log \left( p(\bvec{z}_1| \bvec{\pi})\prod_{n=2}^Np(\bvec{z}_n|\bvec{z}_{n-1},A)\prod_{n=1}^Np(\bvec{x}_n|\bvec{z}_n,\bvec{\phi}) \right) \\
    & =  \sum_Zp(Z| \theta^{old})\left(\log p(\bvec{z}_1|\bvec{\pi}) + \sum_{n=2}^N\log p(\bvec{z}_n|\bvec{z}_{n-1},A) + \sum_{n=1}^N\log p(\bvec{x}_n|\bvec{z}_n,\bvec{\phi})\right) \\
    & =  \sum_Zp(Z| \theta^{old})\log p(\bvec{\bvec{z}_1|\bvec{\pi}}) + \sum_Zp(Z| \theta^{old})\sum_{n=2}^N\log p(\bvec{z}_n|\bvec{z}_{n-1},A) + \sum_Zp(Z| \theta^{old})\sum_{n=1}^N\log p(\bvec{x}_n|\bvec{z}_n,\bvec{\phi}) \label{eq:q-org}
  %% \end{aligned}
\end{align}
となる.さて,$p(Z|X,\theta^{old})$について,$Z$が1次マルコフモデルであることに気をつけると,
\begin{align}
  p(Z|X,\theta^{old}) &=  p(\bvec{z}_1, ..., \bvec{z}_N|X,\theta^{old}) \\
  &=  p(\bvec{z}_1|X,\theta^{old})p(\bvec{z}_2, ...., \bvec{z}_N|\bvec{z}_1,X, \theta^{old}) \\
  &\vdots \\
  &=  p(\bvec{z}_1|X,\theta^{old})\prod_{n=2}^Np(\bvec{z}_n|\bvec{z}_{n-1},X,\theta^{old}) \label{eq:latent-prod-description}
\end{align}
と変形できる.
 $\sum_Zp(Z|X,\theta^{old}) = 1$より,
\begin{align}
  &\sum_{\bvec{z}_1}p(\bvec{z}_1|X,\theta^{old})\sum_{\bvec{z}_2}p(\bvec{z}_2|\bvec{z}_1=\bvec{z}_1',X,\theta^{old})\dots \sum_{\bvec{z}_N}p(\bvec{z}_N|\bvec{z}_{N-1}=\bvec{z}_{N-1}',X,\theta^{old}) = 1 \label{eq:z-prod}
\end{align}
となる.ここで,式\eqref{eq:z-prod}の各項は
確率密度関数の定義から$1$である.

以上より,
\eqref{eq:latent-prod-description},\eqref{eq:z-prod}
から,\eqref{eq:q-org}の各項は,
\begin{align}
  & \sum_Zp(Z|X,\theta^{old})\log p(\bvec{z}_1,\pi) \\
  &= \sum_{\bvec{z}_1}p(\bvec{z}_1|X,\theta^{old})\log p(\bvec{z}_1,\pi)\sum_{\bvec{z}_2,\dots \bvec{z}_N}p(\bvec{z}_n|\bvec{z}_{n-1},X,\theta) \\
  &= \sum_{\bvec{z}_1}p(\bvec{z}_1|X,\theta^{old})\log p(\bvec{z}_1,\pi) \label{eq:initial}
\end{align}

\begin{align}
  \sum_Zp(Z| \theta^{old})\sum_{n=2}^N\log p(\bvec{z}_n|\bvec{z}_{n-1},A) &= \sum_Zp(Z| \theta^{old})\{\log p(\bvec{z}_2|\bvec{z}_1) + \dots + \log p(\bvec{z}_N|\bvec{z}_{N-1})\} \\
  &= \sum_Zp(Z| \theta^{old})\log p(\bvec{z}_2|\bvec{z}_1) + \dots + \sum_Zp(Z| \theta^{old})\log p(\bvec{z}_N|\bvec{z}_{N-1}) \\
  &= \sum_{n=2}^N\sum_Zp(Z|\theta^{old})\log p(\bvec{z}_n|\bvec{z}_{n-1}) \\
  &= \sum_{n=2}^N\sum_{\bvec{z}_n,\bvec{z}_{n-1}}p(\bvec{z}_n,\bvec{z}_{n-1})\log p(\bvec{z}_n|\bvec{z}_{n-1}) \\
  &\hspace{10mm}\sum_{\bvec{z}_1,\dots,\bvec{z}_{n-2},\bvec{z}_{n+1},\dots,\bvec{z}_N}p(\bvec{z}_1,\dots,\bvec{z}_{n-2},\bvec{z}_{n+1},\dots,\bvec{z}_N|\bvec{z}_n,\bvec{z}_{n-1},\theta^{old}) \\
  &= \sum_{n=2}^N\sum_{\bvec{z}_n,\bvec{z}_{n-1}}p(\bvec{z}_n,\bvec{z}_{n-1}|\theta^{old})\log p(\bvec{z}_n|\bvec{z}_{n-1}) \label{eq:transition}\\
\end{align}
\begin{align}
  & \sum_Zp(Z|X,\theta^{old})\sum_{n=1}^N\log p(\bvec{x}_n|\bvec{z}_n,\bvec{\phi}) \\
  &= \sum_{n=1}^N\sum_{\bvec{z}_n}p(\bvec{z}_n|X,\theta^{old})\log p(\bvec{x}_n|\bvec{z}_n,\bvec{\phi})\sum_{\bvec{z}_1,\dots,\bvec{z}_{n-1},\bvec{z}_{n+1},\dots,\bvec{z}_N}p(\bvec{z}_1,\dots,\bvec{z}_{n-1},\bvec{z}_{n+1},\dots,\bvec{z}_N|\bvec{z}_n,X,\theta^{old}) \\
  &= \sum_{n=1}^N\sum_{\bvec{z}_n}p(\bvec{z}_n|X,\theta^{old})\log p(\bvec{x}_n|\bvec{z}_n,\bvec{\phi}) \label{eq:emission}
\end{align}

のように変形できる.

\eqref{eq:initial},\eqref{eq:transition},
\eqref{eq:emission}を\eqref{eq:q-org}に代入すると,
\begin{align}
  Q(\theta, \theta^{old}) &=  \sum_Zp(Z| \theta^{old})\log p(X, Z| \theta) \\
  &= \sum_{\bvec{z}_1}p(\bvec{z}_1|X,\theta^{old})\log p(\bvec{z}_1,\pi) + \sum_{n=2}^N\sum_{\bvec{z}_n,\bvec{z}_{n-1}}p(\bvec{z}_n,\bvec{z}_{n-1}|\theta^{old})\log p(\bvec{z}_n|\bvec{z}_{n-1},A) \\
  &{}\hspace{20mm} + \sum_{n=1}^N\sum_{\bvec{z}_n}p(\bvec{z}_n|X,\theta^{old})\log p(\bvec{x}_n|\bvec{z}_n,\bvec{\phi}) \label{eq:q-parse}
\end{align}
式\eqref{eq:q-parse}の各項について,
式\eqref{eq:initial-probability},
\eqref{eq:transition-probability}に気をつけながら
潜在変数の期待値についての表現に直すと,
\begin{align}
  &\sum_{\bvec{z}_1}p(\bvec{z}_1|X,\theta^{old})\log p(\bvec{z}_1|\bvec{\pi}) \\
  &= \sum_{\bvec{z}_1}\gamma(\bvec{z}_1)\log \sum_{k=1}^K\pi_k^{z_{1k}} \\
  &= \gamma\left(\begin{bmatrix} 1 \\ 0 \\ \vdots \\ 0 \\ \end{bmatrix}\right) \log \sum_{k=1}^K\pi_k^{z_{1k}} + \dots + \gamma\left(\begin{bmatrix} 0 \\ \vdots \\ 0 \\ 1 \\ \end{bmatrix}\right) \log \sum_{k=1}^K\pi_k^{z_{1k}} \\
  &= \gamma(z_{11})\log \pi_1 + \dots + \gamma(z_{1K})\log \pi_K \\
  &= \sum_{k=1}^K\gamma(z_{1k})\log \pi_k \label{eq:initial-expect} \\\\
  & \sum_{n=2}^N\sum_{\bvec{z}_n,\bvec{z}_{n-1}}p(\bvec{z}_n,\bvec{z}_{n-1}|\theta^{old})\log p(\bvec{z}_n|\bvec{z}_{n-1}) \\
  &= \sum_{n=2}^N\sum_{\bvec{z}_n,\bvec{z}_{n-1}}\xi(\bvec{z}_n,\bvec{z}_{n-1})\log \prod_{k=1}^K\prod_{j=1}^KA_{jk}^{z_{n-1,j}z_{nk}} \\
  %\log p(\bvec{z}_n|\bvec{z}_{n-1}) \\
  &= \sum_{n=2}^N\xi\left(\begin{bmatrix} 1 \\ 0 \\ \vdots \\ 0 \\ \end{bmatrix},\begin{bmatrix} 1 \\ 0 \\ \vdots \\ 0 \\ \end{bmatrix}\right)\log A_{11} + \dots + \xi\left(\begin{bmatrix} 1 \\ 0 \\ \vdots \\ 0 \\ \end{bmatrix},\begin{bmatrix} 0 \\ 0 \\ \vdots \\ 1 \\ \end{bmatrix}\right)\log A_{1K} + \dots + \xi\left(\begin{bmatrix} 0 \\ 0 \\ \vdots \\ 1 \\ \end{bmatrix},\begin{bmatrix} 0 \\ 0 \\ \vdots \\ 1 \\ \end{bmatrix}\right)\log A_{KK} \\
  &= \sum_{n=2}^N\sum_{j=1}^K\sum_{k=1}^K\xi(z_{n-1,j},z_{nk})\log A_{jk} \label{eq:transition-expect} \\\\
  &\sum_{n=1}^N\sum_{\bvec{z}_n}p(\bvec{z}_n|X,\theta^{old})\log p(\bvec{x}_n|\bvec{z}_n,\bvec{\phi}) \\
  &= \sum_{n=1}^N\sum_{k=1}^K\gamma(z_{nk})\log p(\bvec{x}_n|\phi_k) \label{eq:emission-expect} \\
\end{align}


\eqref{eq:initial-expect},\eqref{eq:transition-expect},\eqref{eq:emission-expect}を\eqref{eq:q-parse}に代入すると,

\begin{align}
  Q(\theta, \theta^{old}) = \sum_{k=1}^K\gamma(z_{1k})\log \pi_k + \sum_{n=2}^N\sum_{j=1}^K\sum_{k=1}^K\xi(z_{n-1,j},z_{nk})\log A_{jk} + \sum_{n=1}^N\sum_{k=1}^K\gamma(z_{nk})\log p(\bvec{x}_n|\phi_k) \label{eq:q-likelihood}
\end{align}

式\eqref{eq:q-likelihood}より,隠れマルコフモデルを表す
確率分布関数の現在のパラメータに関する期待値を
潜在変数が出力される期待値である
$\gamma(z_{nk})$,$\xi(z_{n-1,j},z_{nk})$
によって表すことができた.

あとはこのモデルの期待値を最大化するパラメータをM stepで求めればよい.

\subsubsection{M step}
\label{sec:Baum-Welch:m-step}
ラグランジュの未定乗数法を用いて,式\eqref{eq:q-likelihood}を
最大化するパラメータを求める.パラメータの制約条件は,
\begin{align}
  &\sum_{k=1}^K\pi_k = 1 \label{eq:pi-constraint}\\
  &\sum_{k=1}^KA_{jk} = 1 (\forall j, j\in \{1,\dots, K\}) \label{eq:A-constraint}
\end{align}
である.よって以下の関数$F(\bvec{\pi},A,\bvec{\phi})$を考えると,
各ラグランジュ未定乗数を$\lambda_i (i\in \{1,\dots,K+1\})$として,
\begin{align}
  F(\bvec{\pi},A,\bvec{\phi}) = Q - \lambda_1\left(\sum_{k=1}^K\pi_k-1\right) -\sum_{j=1}^K\left\{\lambda_{j+1}\left(\sum_{k=1}^KA_{jk}-1\right)\right\}
\end{align}
という式で表すことができる.この関数を各パラメータに対して最大化する.

まず,$\bvec{\pi}$について考えると,
\begin{align}
  \frac{\partial F}{\partial \pi_k} &= \frac{\partial Q}{\partial \pi_k} - \lambda_1 \\
  &= \frac{\gamma(z_{1k})}{\pi_k} - \lambda_1= 0\\
\end{align}
より,
\begin{align}
  \pi_k = \frac{\gamma(z_{1k})}{\lambda_1}
\end{align}
となる.\eqref{eq:pi-constraint}から,
\begin{align}
  &\sum_{k=1}^K\pi_k = 1 \\
  &\Leftrightarrow \sum_{k=1}^K\frac{\gamma(z_{1k})}{\lambda_1} = 1 \\
  &\Leftrightarrow \lambda_1 = \sum_{k=1}^K\gamma(z_{1k})
\end{align}
以上より,
\begin{align}
  \pi_k = \frac{\gamma(z_{1k})}{\sum_{k=1}^K\gamma(z_{1k})} \label{eq:pi-max}
\end{align}
となる.
%% 以下の式は間違っている可能性が大きい
%% ここで,
%% $\gamma(\bvec{z}_n)=p(\bvec{z}_n|X,\theta^{old})$より,
%% \begin{align}
%%   \gamma(\bvec{z}_n) &= \sum_{k=1}^Kp(z_{nk}=1|X,\theta^{old}) \\
%%   &= \sum_{k=1}^K\sum_{\bvec{z}_n}p(\bvec{z}_n|X,\theta^{old})z_{nk} &\hspace{10mm}(z_{nk}=1)\\
%%   &= \sum_{k=1}^K\sum_{\bvec{z}_n}\gamma(\bvec{z}_n)z_{nk} &\hspace{10mm}(z_{nk}=1)\\
%%   &= \sum_{k=1}^K\gamma(z_{nk})\\
%% \end{align}
%% である.\\

次に$A$について考える.
\begin{align}
  \frac{\partial F}{\partial A_{jk}} &= \sum_{n=2}^N\xi(z_{n-1,j},z_{nk})\frac{1}{A_{jk}} - \lambda_{j+1} = 0 \\
  %% &= \frac{\gamma(z_{nk})}{A_{jk}} - \lambda_{j+1}= 0\\
\end{align}
より,
\begin{align}
  &A_{jk} = \frac{\sum_{n=2}^N\xi(z_{n-1,j},z_{nk})}{\lambda_{j+1}}  \\
  %% \frac{\gamma(z_{nk})}{\lambda_{j+1}}
\end{align}
となる.\eqref{eq:A-constraint}から,
\begin{align}
  &\sum_{k=1}^KA_{jk} = 1 \\
  &\sum_{k=1}^K\frac{\sum_{n=2}^N\xi(z_{n-1,j},z_{nk})}{\lambda_{j+1}} = 1 \\
  &\Leftrightarrow \lambda_{j+1} = \sum_{k=1}^K\sum_{n=2}^N\xi(z_{n-1,j},z_{nk}) \\
\end{align}
以上より,
\begin{align}
  A_{jk} = \frac{\sum_{n=2}^N\xi(z_{n-1,j},z_{nk})}{\sum_{k=1}^K\sum_{n=2}^N\xi(z_{n-1,j},z_{nk})} \label{eq:A-max}
\end{align}
となる.
%% 以下の式は間違っている可能性が大きい
%% $\xi(z_{n-1,j},z_{nk})$については先ほどと同様に,
%% \begin{align}
%%   \xi(\bvec{z}_{n-1},\bvec{z}_n) = \sum_{k=1}^K\sum_{n=2}^N\xi(z_{n-1,j},z_{nk})
%% \end{align}
%% となる.

最後に,$\bvec{\phi}$に関して$F$を最大化する.
$\phi$に関しては出力の確率分布関数がどのような分布を仮定するかによるので,
ここでは正規分布のときの例を考える.すなわち,
\begin{align}
  &p(\bvec{x}_n|\phi_k) = \mathcal{N}(\bvec{x}|\bvec{\mu},\Sigma_k) \\
  &\hphantom{p(\bvec{x}_n|\phi_k)}= \frac{1}{(2\pi)^{\frac{D}{2}}}\frac{1}{|\Sigma_k|^{\frac{1}{2}}}\exp \{-\frac{1}{2}(\bvec{x}-\bvec{\mu}_k)^{\mathrm{T}}\Sigma_k^{-1}(\bvec{x}-\bvec{\mu}_k)\} \\
  &\Leftrightarrow \log p(\bvec{x}_n|\phi_k) = C - \frac{1}{2}\log |\Sigma_k| - \frac{1}{2}(\bvec{x}-\bvec{\mu}_k)^{\mathrm{T}}\Sigma_k^{-1}(\bvec{x}-\bvec{\mu}_k) \hspace{10mm}(C:定数)
\end{align}
よって,
\begin{align}
  \displaystyle &\frac{\partial F}{\partial \mu_k} = \sum_{n=1}^N\gamma(z_{nk})\left[(\bvec{x}_n-\bvec{\mu}_k)\{\Sigma_k^{-1} + (\Sigma_k^{-1})^{\mathrm{T}}\}\right] = 0\\
  &\Leftrightarrow \{\Sigma_k^{-1} + (\Sigma_k^{-1})^{\mathrm{T}}\}\sum_{n=1}^N\gamma(z_{nk})(\bvec{x}_n-\bvec{\mu}_k) = 0 \\
  &\Leftrightarrow \bvec{\mu}_k = \frac{\sum_{n=1}^N\gamma(z_{nk})\bvec{x}_n}{\sum_{n=1}^N\gamma(z_{nk})} \label{eq:mu-max}
\end{align}
また,
\begin{align}
  % &\frac{\partial F}{\partial \Sigma_k} = ?? \\
  &\frac{\partial F}{\partial \Sigma_k} = \frac{\partial}{\partial \Sigma_k} \sum_{n=1}^{N}\sum_{k'=1}^{K}\gamma(z_{nk'}) \left\{ C - \frac{1}{2}\log |\Sigma_{k'}| - \frac{1}{2}(\bvec{x}-\bvec{\mu})^\mathrm{T}\Sigma_{k'}^{-1}(\bvec{x}-\bvec{\mu}) \right\} = 0\\
  &\Leftrightarrow \sum_{n=1}^{N}\gamma(z_{nk}) \frac{\partial}{\partial \Sigma_k}\log |\Sigma_k| + \sum_{n=1}^{N}\gamma(z_{nk}) \frac{\partial}{\partial \Sigma_k} (\bvec{x}-\bvec{\mu})^\mathrm{T}\Sigma_{k}^{-1}(\bvec{x}-\bvec{\mu}) = 0 \label{eq:sigma-max-pre}\\
\end{align}

ここで,任意の行列$A$とベクトル$\bvec{x}$について,
\begin{align}
  \displaystyle \frac{\partial}{\partial A} \log |A| = \left ( A^{-1}\right )^\mathrm{T}
\end{align}

と,

\begin{align}
  \displaystyle \frac{\partial}{\partial A} \left ( \bvec{x}^\mathrm{T}A^{-1}\bvec{x}\right ) &= \frac{\partial}{\partial A}\mathrm{trace}\left ( A^{-1} \bvec{x} \bvec{x}^\mathrm{T} \right ) \\
  &= -A^{-1} \bvec{x} \bvec{x}^\mathrm{T} A^{-1}
\end{align}

である事実を利用すると\footnotemark,\eqref{eq:sigma-max-pre}は,
\begin{align}
  &\sum_{n=1}^{N}\gamma(z_{nk})  \Sigma_k^{-1} - \sum_{n=1}^{N}\gamma(z_{nk}) \Sigma_k^{-1}  \left (\bvec{x}_n - \bvec{\mu}_k \right ) \left ( \bvec{x}_n - \bvec{\mu}_k \right ) ^\mathrm{T} \Sigma_k^{-1} = 0 \\
  &\Leftrightarrow \Sigma_k=\frac{\sum_{n=1}^N\gamma(z_{nk})(\bvec{x}_n-\bvec{\mu}_k)(\bvec{x}_n-\bvec{\mu}_k)^{\mathrm{T}}}{\sum_{n=1}^N\gamma(z_{nk})} \label{eq:sigma-max}
\end{align}

\footnotetext{PRMLの確率分布の章の練習問題の解答にもう少し細かい式展開があったはず.}

であることがわかる.なお,$\Sigma_k^{-1}$は精度行列と呼ばれるものであり,対称行列である
ことから$\left (\Sigma_k^{-1} \right )^{\mathrm{T}} = \Sigma_k^{-1}$であることに注意する.

\eqref{eq:pi-max},\eqref{eq:A-max},\eqref{eq:mu-max},\eqref{eq:sigma-max} と,後述するForward-Backward アルゴリズムを用いて
$\gamma(\bvec{z}_n)$と$\xi(\bvec{z}_{n-1},\bvec{z}_n)$
について計算すれば
モデルの期待値を陽に求めることができる.

%% EOF

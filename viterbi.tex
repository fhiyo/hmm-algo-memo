\section{Viterbi algorithm}
\label{sec:Viterbi}
\subsection{概要}
\label{sec:Viterbi:abst}
先ほどのForward-Backwardアルゴリズムでは,
各潜在変数についての事後確率を求めた.
しかし,各々の潜在変数について,
観測データ列が与えられたときの事後確率が最大となるような
潜在変数の列$\bvec{z}_1,\dots,\bvec{z}_N$を推定することは
できない.
%% 観測変数の列$\bvec{x}_1,\dots,\bvec{x}_N$が出力されるときの
%% モデルの尤度関数を最大化したいときに
%% Forward-Backwardアルゴリズムを適用することはできない.
そこで提案されるのがViterbiアルゴリズムである.
このアルゴリズムはある観測データ列
$\bvec{x}_1,\dots,\bvec{x}_N$が出力されたときの
潜在変数の列$\bvec{z}_1,\dots,\bvec{z}_N$
の事後確率を最大化するアルゴリズムである.
\subsubsection{入力}
\label{sec:Viterbi:input}
%% 隠れマルコフモデルの確率分布$p(X,Z)$,
パラメータ$\theta = \{\bvec{\pi}, A, \bvec{\phi}\}$ \\
\hphantom{観}観測データ列$\bvec{x}_1,\dots,\bvec{x}_N$
\subsubsection{出力}
\label{sec:Viterbi:output}
事後確率$p(Z|X)$を最大化する
潜在変数の列$\bvec{z}_1,\dots,\bvec{z}_N$ \\
\hphantom{潜}任意の時点$i$における最大事後確率
$p(\bvec{z}_1,\dots,\bvec{z}_i|X,\theta)$
\subsection{アルゴリズム}
\label{sec:Viterbi:algorithm}
以下の式で与えられる$\omega(\bvec{z}_n)$を考える.
\begin{align}
  \omega(\bvec{z}_n) =
  \begin{cases}
    \max_{\bvec{z}_1,\dots,\bvec{z}_{n-1}}\log p(\bvec{x}_1,\dots,\bvec{x}_n,\bvec{z}_1,\dots,\bvec{z}_n)\hspace{10mm}&(n > 1) \label{eq:Viterbi:omega-induction}\\
    \hspace{5mm}\omega(\bvec{z}_1) = \log (\bvec{x}_1,\bvec{z}_1)\hspace{5mm}&(n = 1)\\
  \end{cases}
  %% &\omega(\bvec{z}_n) = \max_{\bvec{z}_1,\dots,\bvec{z}_{n-1}}\log p(\bvec{x}_1,\dots,\bvec{x}_n,\bvec{z}_1,\dots,\bvec{z}_n) \label{eq:Viterbi:omega-induction}\\
  %% &\omega(\bvec{z}_1) = \log (\bvec{x}_1,\bvec{z}_1) \\
\end{align}

以下,$\omega(\bvec{z}_n)$についての漸化式を解く.
式\eqref{eq:Viterbi:omega-induction}より,
\begin{align}
  \omega(\bvec{z}_{n+1}) &= \max_{\bvec{z}_1,\dots,\bvec{z}_n}\log p(\bvec{x}_1,\dots,\bvec{x}_{n+1},\bvec{z}_1,\dots,\bvec{z}_{n+1}) \\
  &= \max_{\bvec{z}_1,\dots,\bvec{z}_n}\log p(\bvec{z}_{n+1})p(\bvec{x}_1,\dots,\bvec{x}_{n+1},\bvec{z}_1,\dots,\bvec{z}_n|\bvec{z}_{n+1}) \\
  &= \max_{\bvec{z}_1,\dots,\bvec{z}_n}\log p(\bvec{x}_{n+1}|\bvec{z}_{n+1})p(\bvec{z}_{n+1})p(\bvec{x}_1,\dots,\bvec{x}_n,\bvec{z}_1,\dots,\bvec{z}_n|\bvec{x}_{n+1},\bvec{z}_{n+1}) \\
  &= \max_{\bvec{z}_1,\dots,\bvec{z}_n}\log p(\bvec{x}_{n+1}|\bvec{z}_{n+1})p(\bvec{z}_{n+1})p(\bvec{x}_1,\dots,\bvec{x}_n,\bvec{z}_1,\dots,\bvec{z}_n|\bvec{z}_{n+1}) \\
  &= \log p(\bvec{x}_{n+1}|\bvec{z}_{n+1})+\max_{\bvec{z}_1,\dots,\bvec{z}_n}\log p(\bvec{z}_{n+1})p(\bvec{x}_1,\dots,\bvec{x}_n,\bvec{z}_1,\dots,\bvec{z}_n|\bvec{z}_{n+1}) \\
  &= \log p(\bvec{x}_{n+1}|\bvec{z}_{n+1})+\max_{\bvec{z}_1,\dots,\bvec{z}_n}\log p(\bvec{z}_{n+1})p(\bvec{x}_1,\dots,\bvec{x}_n,\bvec{z}_1,\dots,\bvec{z}_n|\bvec{z}_{n+1}) \label{eq:omega-on-the-way}\\
\end{align}
ここで,$P = p(\bvec{z}_{n+1})p(\bvec{x}_1,\dots,\bvec{x}_n,\bvec{z}_1,\dots,\bvec{z}_n|\bvec{z}_{n+1})$
とおくと,
\begin{align}
  P &= p(\bvec{z}_{n+1})p(\bvec{x}_1,\dots,\bvec{x}_n,\bvec{z}_1,\dots,\bvec{z}_n|\bvec{z}_{n+1}) \\
  &= p(\bvec{x}_1,\dots,\bvec{x}_n,\bvec{z}_1,\dots,\bvec{z}_n,\bvec{z}_{n+1}) \\
  &= p(\bvec{z}_n)p(\bvec{x}_1,\dots,\bvec{x}_n,\bvec{z}_1,\dots,\bvec{z}_{n+1}|\bvec{z}_n) \\
  &= p(\bvec{z}_n)p(\bvec{x}_1,\dots,\bvec{x}_n,\bvec{z}_1,\dots|\bvec{z}_{n+1},\bvec{z}_n)p(\bvec{z}_{n+1}|\bvec{z}_n) \\
  &= p(\bvec{z}_n)p(\bvec{x}_1,\dots,\bvec{x}_n,\bvec{z}_1,\dots|\bvec{z}_n)p(\bvec{z}_{n+1}|\bvec{z}_n) \\
  &= p(\bvec{x}_1,\dots,\bvec{x}_n,\bvec{z}_1,\dots,\bvec{z}_n)p(\bvec{z}_{n+1}|\bvec{z}_n) \label{eq:omega-p}\\
\end{align}
と変形できる.式\eqref{eq:omega-p}を
式\eqref{eq:omega-on-the-way}に代入すると,
\begin{align}
  \omega(\bvec{z}_{n+1}) &= \log p(\bvec{x}_{n+1}|\bvec{z}_{n+1})+\max_{\bvec{z}_1,\dots,\bvec{z}_n}\log P \\
  &= \log p(\bvec{x}_{n+1}|\bvec{z}_{n+1})+\max_{\bvec{z}_1,\dots,\bvec{z}_n}\log  p(\bvec{x}_1,\dots,\bvec{x}_n,\bvec{z}_1,\dots,\bvec{z}_n)p(\bvec{z}_{n+1}|\bvec{z}_n) \\
  \displaystyle &= \log p(\bvec{x}_{n+1}|\bvec{z}_{n+1})+\max_{\bvec{z}_n}\left\{\max_{\bvec{z}_1,\dots,\bvec{z}_{n-1}}\log  p(\bvec{x}_1,\dots,\bvec{x}_n,\bvec{z}_1,\dots,\bvec{z}_n)p(\bvec{z}_{n+1}|\bvec{z}_n)\right\} \\
  \displaystyle &= \log p(\bvec{x}_{n+1}|\bvec{z}_{n+1})+\max_{\bvec{z}_n}\left\{\log p(\bvec{z}_{n+1}|\bvec{z}_n)+\max_{\bvec{z}_1,\dots,\bvec{z}_{n-1}}\log  p(\bvec{x}_1,\dots,\bvec{x}_n,\bvec{z}_1,\dots,\bvec{z}_n)\right\} \\
  \displaystyle &= \log p(\bvec{x}_{n+1}|\bvec{z}_{n+1})+\max_{\bvec{z}_n}\left\{\log p(\bvec{z}_{n+1}|\bvec{z}_n)+\omega(\bvec{z}_n)\right\} \label{eq:Viterbi:omega-recursion}\\
\end{align}
となる.

式\eqref{eq:Viterbi:omega-recursion}を$\omega(\bvec{z}_1)$から順に繰り返し計算することで
$\omega(\bvec{z}_N)$を求めたら,
$\displaystyle \bvec{z}_N' = \arg\max_{\bvec{z}_N}\omega(\bvec{z}_N)$を求め,
このときの$\bvec{z}_N'$を出力する.
つぎに,
\begin{align}
  &\omega(\bvec{z}_N') = \log p(\bvec{x}_N|\bvec{z}_N')+\max_{\bvec{z}_{N-1}}\left\{\log p(\bvec{z}_N'|\bvec{z}_{N-1})+\omega(\bvec{z}_{N-1})\right\} \\
\end{align}
の式から,
\begin{align}
  \bvec{z}_{N-1}' = \arg\max_{\bvec{z}_{N-1}}\left\{\log p(\bvec{z}_N'|\bvec{z}_{N-1})+\omega(\bvec{z}_{N-1})\right\}
\end{align}
となるような$\bvec{z}_{N-1}'$を求めることができる.
以下これを$\bvec{z}_1'$まで繰り返し求めることで,
観測データ列が与えられたときの事後確率を最大化するような
潜在変数の列$\bvec{z}_1',\dots,\bvec{z}_N'$
を求めることができる.

また,計算の過程で任意の時点$i$における最大事後確率
$p(\bvec{z}_1,\dots,\bvec{z}_i|X,\theta)$
を求めることが出来る.

%% EOF

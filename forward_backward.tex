\section{Forward-Backward algorithm}
\label{sec:Forward-Backward}
\subsection{概要}
\label{sec:Forward-Backward:abst}
モデルの各潜在変数の事後確率である
$\gamma(\bvec{z}_n)$と$\xi(\bvec{z}_{n-1},\bvec{z}_n)$
を求めるためのアルゴリズムである.
観測変数$X$を取得したときの潜在変数の推定ができる.

なおsection \ref{sec:Forward-Backward},
及びsection \ref{sec:Viterbi}
では確率分布のパラメータ依存性を式に明記しないことにする.
\subsubsection{入力}
\label{sec:Forward-Backward:input}
%% HMMの確率分布関数$p(X,Z|\theta)$,
観測データ列 $X=\{\bvec{x}_1,\dots,\bvec{x}_N\}$,
パラメータ $\theta^{old} = \{\bvec{\pi}, A, \bvec{\phi}\}$
\subsubsection{出力}
\label{sec:Forward-Backward:output}
$\gamma(\bvec{z}_n)$と$\xi(\bvec{z}_{n-1},\bvec{z}_n)$
の値
\subsection{アルゴリズム}
\label{sec:Forward-Backward:algorithm}
Bayes' theoremを用いると,$\gamma(\bvec{z}_n)$は
\begin{align}
  \gamma(\bvec{z}_n) = p(\bvec{z}_n|X) = \frac{p(X|\bvec{z}_n)p(\bvec{z}_n)}{p(X)}
\end{align}
と表せる.ここで,任意の観測変数どうしは独立なので,
\begin{align}
  \gamma(\bvec{z}_n) &= \frac{p(X|\bvec{z}_n)p(\bvec{z}_n)}{p(X)} \\
  &= \frac{p(\bvec{x}_1,\dots,\bvec{x}_N|\bvec{z}_n)p(\bvec{z}_n)}{p(X)} \\
  &= \frac{p(\bvec{x}_1,\dots,\bvec{x}_n|\bvec{z}_n)p(\bvec{x}_{n+1},\dots,\bvec{x}_N|\bvec{x}_1,\dots,\bvec{x}_n,\bvec{z}_n)p(\bvec{z}_n)}{p(X)} \\
  &= \frac{p(\bvec{x}_1,\dots,\bvec{x}_n,\bvec{z}_n)p(\bvec{x}_{n+1},\dots,\bvec{x}_N|\bvec{x}_1,\dots,\bvec{x}_n,\bvec{z}_n)}{p(X)} \\
  &= \frac{p(\bvec{x}_1,\dots,\bvec{x}_n,\bvec{z}_n)p(\bvec{x}_{n+1},\dots,\bvec{x}_N|\bvec{z}_n)}{p(X)} \\
  &= \frac{\alpha(\bvec{z}_n)\beta(\bvec{z}_n)}{p(X)} \\
  &= \frac{\alpha(\bvec{z}_n)\beta(\bvec{z}_n)}{\sum_{\bvec{z}_n}\alpha(\bvec{z}_n)\beta(\bvec{z}_n)} \label{eq:fwbw-gamma} \\
\end{align}
と表すことが出来る.ただし,
\begin{align}
  &\alpha(\bvec{z}_n)\defeq
  \begin{cases}
    \displaystyle \prod_{k=1}^K\left\{\pi_kp(\bvec{x}_1|\phi_k)\right\}^{z_{nk}}&\hspace{10mm} (n = 1)\\
    p(\bvec{x}_1,\dots,\bvec{x}_n,\bvec{z}_n)&\hspace{10mm} (n\in \{2,\dots,N\})
  \end{cases} \\\\
    &\beta(\bvec{z}_n)\defeq
  \begin{cases}
    1&\hspace{10mm} (n = N)\\
    p(\bvec{x}_{n+1},\dots,\bvec{x}_N|\bvec{z}_n)&\hspace{10mm} (n\in \{1,\dots,N-1\})\\
  \end{cases}
\end{align}
とする.このように定義すると,$\alpha(\bvec{z}_n)$,$\beta(\bvec{z}_n)$のそれぞれについて漸化式を立てることができ,
\begin{align}
  \alpha(\bvec{z}_n) &= p(\bvec{x}_1,\dots,\bvec{x}_n,\bvec{z}_n) \\
  &= p(\bvec{z}_n)p(\bvec{x}_1,\dots,\bvec{x}_n|\bvec{z}_n) \\
  &= p(\bvec{z}_n)p(\bvec{x}_n|\bvec{z}_n)p(\bvec{x}_1,\dots,\bvec{x}_{n-1}|\bvec{x}_n,\bvec{z}_n) \\
  &= p(\bvec{z}_n)p(\bvec{x}_n|\bvec{z}_n)p(\bvec{x}_1,\dots,\bvec{x}_{n-1}|\bvec{z}_n) \\
  &= p(\bvec{x}_n|\bvec{z}_n)p(\bvec{x}_1,\dots,\bvec{x}_{n-1},\bvec{z}_n) \\
  &= p(\bvec{x}_n|\bvec{z}_n)\sum_{\bvec{z}_{n-1}}p(\bvec{x}_1,\dots,\bvec{x}_{n-1},\bvec{z}_{n-1},\bvec{z}_n) \\
  &= p(\bvec{x}_n|\bvec{z}_n)\sum_{\bvec{z}_{n-1}}p(\bvec{z}_{n-1},\bvec{z}_n)p(\bvec{x}_1,\dots,\bvec{x}_{n-1}|\bvec{z}_{n-1},\bvec{z}_n) \\
  &= p(\bvec{x}_n|\bvec{z}_n)\sum_{\bvec{z}_{n-1}}p(\bvec{z}_{n-1},\bvec{z}_n)p(\bvec{x}_1,\dots,\bvec{x}_{n-1}|\bvec{z}_{n-1}) \\
  &= p(\bvec{x}_n|\bvec{z}_n)\sum_{\bvec{z}_{n-1}}p(\bvec{z}_{n-1})p(\bvec{z}_n|\bvec{z}_{n-1})p(\bvec{x}_1,\dots,\bvec{x}_{n-1}|\bvec{z}_{n-1}) \\
  &= p(\bvec{x}_n|\bvec{z}_n)\sum_{\bvec{z}_{n-1}}p(\bvec{x}_1,\dots,\bvec{x}_{n-1},\bvec{z}_{n-1})p(\bvec{z}_n|\bvec{z}_{n-1}) \\
  &= p(\bvec{x}_n|\bvec{z}_n)\sum_{\bvec{z}_{n-1}}\alpha(\bvec{z}_{n-1})p(\bvec{z}_n|\bvec{z}_{n-1}) \label{eq:alpha-recursive}
\end{align}

$\beta$についても同様にして計算すると,
\begin{align}
  \beta(\bvec{z}_n) &= p(\bvec{x}_{n+1},\dots,\bvec{x}_N|\bvec{z}_n) \\
  &= \sum_{\bvec{z}_{n+1}}p(\bvec{x}_{n+1},\dots,\bvec{x}_N,\bvec{z}_{n+1}|\bvec{z}_n) \\
  &= \sum_{\bvec{z}_{n+1}}p(\bvec{z}_{n+1}|\bvec{z}_n)p(\bvec{x}_{n+1},\dots,\bvec{x}_N|\bvec{z}_{n+1},\bvec{z}_n) \\
  &= \sum_{\bvec{z}_{n+1}}p(\bvec{z}_{n+1}|\bvec{z}_n)p(\bvec{x}_{n+1},\dots,\bvec{x}_N|\bvec{z}_{n+1}) \\
  &= \sum_{\bvec{z}_{n+1}}p(\bvec{z}_{n+1}|\bvec{z}_n)p(\bvec{x}_{n+1}|\bvec{z}_{n+1})p(\bvec{x}_{n+2},\dots,\bvec{x}_N|\bvec{x}_{n+1},\bvec{z}_{n+1}) \\
  &= \sum_{\bvec{z}_{n+1}}p(\bvec{x}_{n+2},\dots,\bvec{x}_N|\bvec{z}_{n+1})p(\bvec{x}_{n+1}|\bvec{z}_{n+1})p(\bvec{z}_{n+1}|\bvec{z}_n) \\
  &= \sum_{\bvec{z}_{n+1}}\beta(\bvec{z}_{n+1})p(\bvec{x}_{n+1}|\bvec{z}_{n+1})p(\bvec{z}_{n+1}|\bvec{z}_n) \label{eq:beta-recursive}
\end{align}
となる.

\eqref{eq:alpha-recursive}, \eqref{eq:beta-recursive}
後は漸化式を解けば$\alpha(\bvec{z}_n)$,\ $\beta(\bvec{z}_n)$
は陽に求まる.次は$\gamma(\bvec{z}_n)$,\ $\xi(\bvec{z}_{n-1},\bvec{z}_n)$を$\alpha,\beta$を使って表すことが出来れば
任意の潜在変数の最尤値が求まり,目的が達成される.

ところで式\eqref{eq:fwbw-gamma}の分母について,
$p(X)=\sum_{\bvec{z}_n}\alpha(\bvec{z}_n)\beta(\bvec{z}_n)$は
任意の$n$について成り立つので,
\begin{align}
  p(X) &= \sum_{\bvec{z}_n}\alpha(\bvec{z}_n)\beta(\bvec{z}_n) \\
  &= \sum_{\bvec{z}_N}\alpha(\bvec{z}_N)\beta(\bvec{z}_N) \\
  &= \sum_{\bvec{z}_N}\alpha(\bvec{z}_N) \\
\end{align}
で計算することが出来る.つまり式\eqref{eq:fwbw-gamma}は
\begin{align}
  \gamma(\bvec{z}_n) = \frac{\alpha(\bvec{z}_n)\beta(\bvec{z}_n)}{\sum_{\bvec{z}_N}\alpha(\bvec{z}_N)} \label{eq:fwbw-gamma-2}
\end{align}
と表せる.次に
$\xi(\bvec{z}_{n-1},\bvec{z}_n)$について考える.
\begin{align}
  \xi(\bvec{z}_{n-1},\bvec{z}_n) &= p(\bvec{z}_{n-1},\bvec{z}_n|X) \\
  &= \frac{p(X|\bvec{z}_{n-1},\bvec{z}_n)p(\bvec{z}_{n-1},\bvec{z}_n)}{p(X)} \\
  &= \frac{p(\bvec{x}_1,\dots,\bvec{x}_n|\bvec{z}_{n-1},\bvec{z}_n)p(\bvec{x}_{n+1},\dots,\bvec{x}_N|\bvec{x}_1,\dots,\bvec{x}_n,\bvec{z}_{n-1},\bvec{z}_n)p(\bvec{z}_{n-1},\bvec{z}_n)}{p(X)} \\
  &= \frac{p(\bvec{x}_1,\dots,\bvec{x}_n|\bvec{z}_{n-1},\bvec{z}_n)p(\bvec{x}_{n+1},\dots,\bvec{x}_N|\bvec{z}_{n-1},\bvec{z}_n)p(\bvec{z}_{n-1},\bvec{z}_n)}{p(X)} \\
  &= \frac{p(\bvec{x}_1,\dots,\bvec{x}_n|\bvec{z}_{n-1},\bvec{z}_n)p(\bvec{x}_{n+1},\dots,\bvec{x}_N|\bvec{z}_n)p(\bvec{z}_{n-1},\bvec{z}_n)}{p(X)} \\
  &= \frac{p(\bvec{x}_1,\dots,\bvec{x}_{n-1}|\bvec{z}_{n-1},\bvec{z}_n)p(\bvec{x}_n|\bvec{z}_{n-1},\bvec{z}_n)p(\bvec{x}_{n+1},\dots,\bvec{x}_N|\bvec{z}_n)p(\bvec{z}_{n-1},\bvec{z}_n)}{p(X)} \\
  &= \frac{p(\bvec{x}_1,\dots,\bvec{x}_{n-1}|\bvec{z}_{n-1})p(\bvec{x}_n|\bvec{z}_n)p(\bvec{x}_{n+1},\dots,\bvec{x}_N|\bvec{z}_n)p(\bvec{z}_{n-1},\bvec{z}_n)}{p(X)} \\
  &= \frac{p(\bvec{x}_1,\dots,\bvec{x}_{n-1},\bvec{z}_{n-1})p(\bvec{x}_n|\bvec{z}_n)p(\bvec{x}_{n+1},\dots,\bvec{x}_N|\bvec{z}_n)p(\bvec{z}_n|\bvec{z}_{n-1})}{p(X)} \\
  &= \frac{\alpha(\bvec{z}_{n-1})p(\bvec{x}_n|\bvec{z}_n)p(\bvec{z}_n|\bvec{z}_{n-1})\beta(\bvec{z}_n)}{p(X)} \\
  &= \frac{\alpha(\bvec{z}_{n-1})p(\bvec{x}_n|\bvec{z}_n)p(\bvec{z}_n|\bvec{z}_{n-1})\beta(\bvec{z}_n)}{\sum_{\bvec{z}_N}\alpha(\bvec{z}_N)} \label{eq:fwbw-xi}\\
\end{align}

\eqref{eq:fwbw-gamma-2}と\eqref{eq:fwbw-xi}より
目標は達成された.

%% \subsubsection{独立変数,従属変数についてのメモ}
%% \label{sec:dependent-independent}

%% EOF
